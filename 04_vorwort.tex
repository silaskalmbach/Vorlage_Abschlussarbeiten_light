\chapter*{Vorwort}
\addcontentsline{toc}{chapter}{Vorwort}
\label{danksagungen}

\begin{description}
	\item[-] Diese Vorlage dient als grober Leitfaden zu Erstellung der Abschlussarbeit. Die Formatierung ist somit nicht zwingend umzusetzen.
	\item[-] Die Formatierung des Deckblattes sollte, soweit möglich, unverändert bleiben. 
	\item[-] Von der Gliederung der Arbeit kann abgewichen werden, solang dieses sinnig begründbar ist.
\end{description}
\vspace{10mm}
Um mit \LaTeX{} zu Arbeiten, kann z.B. die Kombination folgende Programme verwendet werden.
\begin{flalign*}
&\text{1) MiKTeX:} &&\text{\url{https://miktex.org/download}}&\\
&\text{2) TeXstudio:} &&\text{\url{https://www.texstudio.org/}}
\end{flalign*}
Alternativ besteht auch die Möglichkeit Online-Dienste zu benutzen, welche mögliche Schwierigkeiten bei der Einrichtung umgehen.

\subsection*{Empfohlene Einstellungen dieser Vorlage}
Für eine problemlose Kompilierung des \LaTeX-Dokumentes ist es notwendig, einige Einstellungen in den Editor zu übernehmen.
\begin{description}
	\item[-] Als Standard Bibliographieprogramm sollte Biber ausgewählt werden
	\item[-] Als Standardcompiler ist LuaLaTeX oder PdfLaTeX zu empfehlen
\end{description}