%------------------------------------------------------------------------------
%Description:  LaTeX Thesis Template main file
%Author:       silas.kalmbach@ilek.uni-stuttgart.de
%Created:      2021-03-10
%------------------------------------------------------------------------------


% ============= Einstellungen zur Arbeit =============

\documentclass[%
paper=a4,
bibliography=totoc,	% Literaturverzeichnis im Inhalt
listof=totoc,		% Abb.- und Tab.verzeichnis im Inhalt
listof=entryprefix, % Abkürzungen Abb./Tab. in den Verzeichnissen
fontsize=11pt, 		% Schriftgröße
twoside=off, 		% Off = PDF-Version, On = Booklet / Book (Gedrucktes Exemplar)
headings=openright	% Einfügen von Leerseiten
]{scrbook}          % Dokumentenklasse: KOMA-Script Book


% ============= Einstellungen zum Kompilieren =============

\def\tikzextern{0} % 0 = Nein; 1 = Ja (TikZ Grafiken extern Speichern)
\def\mintedinstall{0} % 0 = Nein; 1 = Ja; Um Minted zu nutzen muss Python und Pygments installiert sein. Alternative: listings

% Variablen einbinden
\input{00_ressources/variables}

% Preambel einbinden
\input{00_ressources/preamble}


% ============= Angaben zur Arbeit =============


\hypersetup{ % Hinterlegte Daten in der PDF
	pdftitle={\TitelDerArbeit},
	pdfsubject={\TypDerArbeit},
	pdfauthor={\StudentNachname, \StudentVorname},
	pdfkeywords={}
}

% ============= Besondere Trennungen ============= 

\hyphenation{De-zi-mal-tren-nung}



%%%%%%%%%%%%%%%%%%%%%%%%%%%%%%%%%%%%%%%%%%%%
% ============= Dokumentbeginn =============
%%%%%%%%%%%%%%%%%%%%%%%%%%%%%%%%%%%%%%%%%%%%

\begin{document}

\pagestyle{empty} %Seiten ohne Kopf- und Fußzeile sowie Seitenzahl
\include{01_titel}
\restoregeometry

\pagestyle{plain.scrheadings} % Leere Kopf- und Fußzeilen
\pagenumbering{Roman} % Römische Seitenzahlen verwenden
\include{02_aufgabenstellung} %->Optional
\include{03_erklaerung}
\include{04_vorwort} %->Optional
\include{05_zusammenfassung}
\tableofcontents %Inhaltsverzeichnis
\cleardoubleoddpage %Beginn auf einer neuen Seite. Bei Doppelseiten rechts
\pagenumbering{arabic}	% Arabische Seitenzahlen verwenden
\include{06_symbole}
\cleardoubleoddpage %Beginn auf einer neuen Seite. Bei Doppelseiten rechts
\pagestyle{scrheadings}	% pagestyle für gesamtes Dokument aktivieren (Kopf- und Fußzeilen)


% ============= Kapitel =============
% durch Eigene Kapitel erstezen

\include{10_bsp} % Zitation
\include{11_bsp} % Programmcode
\include{12_bsp} % Tabellen
\include{13_bsp} % Formeln
\include{14_bsp} % Grafiken
\include{15_bsp} % Anhang

\ifluatex
\chapter{Programmierungen}\label{chap:programming}
\section{{\TeX} if/ else}\label{sec:tex}
Grundlegender Aufbau:\\
\begin{lstlisting}[language=TeX]
\if <token-1><token-2> <tex-code-1> [\else <tex-code-2>] \fi
\end{lstlisting}
Bsp1:\\
\begin{lstlisting}[language=TeX]
\ifx\mycmd\undefined
	undefed
\else
	\if\mycmd1
		defed, 1
	\else
		defed
	\fi
\fi
\end{lstlisting}

\ifx\mycmd\undefined
	undefed
\else
	\if\mycmd1
		defed, 1
	\else
		defed
	\fi
\fi
\\Bsp2:\\
\begin{lstlisting}[language=TeX]
\def\mycmd{1}

\ifx\mycmd\undefined
	undefed
\else
	\if\mycmd1
		defed, 1
	\else
		defed
	\fi
\fi
	
\end{lstlisting}

\def\mycmd{1}

\ifx\mycmd\undefined
	undefed
\else
	\if\mycmd1
		defed, 1
	\else
		defed
	\fi
\fi
\\Bsp3:\\
\begin{lstlisting}[language=TeX]
\def\mycmd{0}

\ifx\mycmd\undefined
	undefed
\else
	\if\mycmd1
		defed, 1
	\else
		defed
	\fi
\fi
\end{lstlisting}


\def\mycmd{0}

\ifx\mycmd\undefined
	undefed
\else
	\if\mycmd1
		defed, 1
	\else
		defed
	\fi
\fi


\section{forPGF}\label{sec:pgf}

\section{Lua}\label{sec:lua}
\begin{lstlisting}[language={[5.0]Lua}]
\count75=1564 % Data existing in the "TeX World"
\directlua{
local x=\number\count75 \space % Transfer TeX data to the "Lua World"
tex.print("x= "..x)
local y = (2*x-65)/5
tex.print(" and y = "..y)
}
\end{lstlisting}

%\count75=1564 % Data existing in the "TeX World"
%\directlua{
%local x=\number\count75 \space % Transfer TeX data to the "Lua World"
%tex.print("x= "..x)
%local y = (2*x-65)/5
%tex.print(" and y = "..y)
%}

Die Kreiszahl $\pi$ hat den Wert \directlua{tex.print(math.pi)}.

\def\aa{tex}
\def\bb{.}
\def\cc{print}
\def\dd{("Hello wie gehts")}
\directlua{\aa\bb\cc\dd}

\directlua{
	tex.print(math.random())
	tex.print(math.random())
}

 % Programmierung
\fi

% ============= Anhang =============

%\appendix
%\begin{appendices}
%	\include{15_bsp}
%\end{appendices}


% ============= Verzeichnise =============

\printbibliography[title=Literaturverzeichnis] %Literaturverzeichnis
\addcontentsline{toc}{chapter}{Literaturverzeichnis}

\listoffigures %Verzeichnis aller Bilder
\addcontentsline{toc}{chapter}{Abbildungsverzeichnis}

\listoftables %Verzeichnis aller Tabellen
\addcontentsline{toc}{chapter}{Tabellenverzeichnis}


% ============= Dokumentende =============

\end{document}
